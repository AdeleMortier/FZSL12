\documentclass[a4paper,10pt]{article}
\usepackage[utf8]{inputenc}
\usepackage[french]{babel}
\usepackage{frbib}
\usepackage{french}

\title{FZSL12 -- Le motif vénitien\\
	Venise et le motif du piège}
\author{Adèle Mortier}

\begin{document}

\maketitle
\nocite{*}

\section*{Introduction}
\section{Une cité envoutante}
	\subsection{Une fascination culturelle}
	D'Annunzio \cite{DAnnunzio1899} : fantasme oriental, impressions de revivre une période révolue : "en ce soir magiquese renouvelassent tous les souffles et tous les mirages de l'Orient lointain".\\
	de Régnier \cite{Regnier1906} : hallucination auditive mêlée au fantasme des représentations mythiques : "[Venise] ne m'avait-elle pas parlé par la voix de Desdémone?"\\
	Melville \cite{Melville1856} : "Comme si le Grand Turc avait installé là son pavillon pour un jour d'été" : fantasme d'Orient, fasciantion et imagination.\\
	Morand \cite{Morand1971} : impssible de se "décharmer" de Venise. Fasciantion par les images de la ville (à distance), au travers d'une aquarelle du père.\\
	\subsection{Une fascination perceptuelle}
		\subsubsection{Perceptions visuelles}
			Taine \cite{Taine1866} : méli-mélo coloré, fascination ("passeait des heures"), soumission ("nature comme maîtresse")\\
			James \cite{James1909} : fascination infinie, "douceur sans fin" à la vue des couleurs \\
			de Régnier \cite{Regnier1924} : on y est attiré par le mystère prodigieux des coins d'ombre où scintillent des lampes dont la lumière a on ne sait quoi de secret et de lointain". Faciantion mystique t religieuse, conscience d'un au-delà.\\
			Suarès \cite{Suares1910} : fascination pour l'or qui pare les bâtiments : "Je flotte dans le rêve de l'or[...]". Réfénrence au mythe de Midas...\\
			Gautier \cite{Gautier1852} : "coup d'oeil vraiment féérique" sur la place Sait-Marc et sa basilique "étrange" et "mystérieuse"\\
	\subsection{Métaphore de la femme}
		"Venise joliefiancée de l'Adriatique"\\
		James \cite{James1909} : Venise est une villedont on "s'éprend". Termefort qui implique soumission.

		
			
			
\section{Une cité de désillusions ?}
	\subsection{Une beauté parfois superficielle}
		James \cite{James1909} : résonancesavec le portrait de Dorian Gray. La peinture reflète la vérité de la ville, la réalité derrière les apparences... "les louables tentatives d'en faire le portrait s'exposent à un résultats livide ou terrifiant". "beauté de surface", caractère superficiel.\\
		Goncourt \cite{Goncourt1855} : description des mosaïques, vision cauchemardesque\\
		Hesse \cite{Hesse1901} : un gout de Venise pour une certaine molesse, un goût essentiellement orienté vers le décorum"\\
	\subsection{Un paysage trop souvent trompeur}
		"Lamort de Venise"\\
		Barbaro \cite{Barbaro1990} : couleurs changeantes de l'eau, métamorphose progressive du milieu\\
		Turner \cite{Turner1843} : "Le pêcheur de Venise déploie sa voile si gaiement peinte, Et ne prend pas garde au démon qui dans un calme menaçant Attend sa proie de chaque soir"\\
		Sartre \cite{Sartre1964} : "couleur de cauchemar de l'eau", référence aux nombreuses intrigues vénitiennes. Venise est une ville traîtresse, qui nous piège et nous enlise ("la chaussée ramollit" pour ralentir la fuite)\\
		Gautier \cite{Gautier1852} : "Les rares réverbères s'y plongeaint en trainées sanglantes, et ses ondes épaisses, noires comme celles du Cocyte, paraisaient étendre leur manteau complaisant sur bien des crimes". Le Cocyte est un fleuve des Enfers.\\
		Sartre \cite{Sartre1952} : "L'eau à Venise n'est pas de l'eau, c'est cent choses à la fois, c'est une bête pustuleuse, une plante vénéneuse, une surface de vitre sur un noir immonde, c'est du pus [...] La ball flottait près de moi, déjà petit cadavre". Eau mortifère.\\
		
	\subsection{Un lieu qui mène à la déchéance}
		"La mort à Venise"\\
		Mann \cite{Mann1947} : Aschenbach personnage déchu physiquement et psychologiquement, victime d'une passion coupable pour un jeune garçon. La maladie de l'homme est le reflet de la maladie de la ville. "C'était Venise, l'insinuante courtisane, la cité qui tient de la légende et du traquennard [...] il se souvient aussi que la ville était malade et s'en cachait par cupidité, etil épiait avec une passion plus effrénée la go,dole qui flottait là-bas devant lui"\\

\section{L'impossible retour en arrière}
	\subsection{Métaphore de l'enlisement}
		Goncourt \cite{Goncourt1855} : "L'eau est engourdie, pâmée, figée, et les mâts jaunes de bateaux et le palais roses s'y reflètent, comme en une huile où les arêtes de lignes se noieraient dans du gras liquide". Motif de l'enlisement. Eau mortelle, sables mouvants\\
		Vircondelet \cite{Vircondelet1979} : "bassin mortel"\\
		Giono \cite{Giono1954} : "le naufrage n'est pas loin" : sentiment de menace permanente, peur d'être définitivement englouti.\\
		Proust \cite{Proust1927} : hypnotisé par le chant des gondoliers. "il aurait fallu décider sans perdre une seconde que je partais, mais c'est justement ce que je ne pouvais pas; je restais immobile, sans être capable non seuelemtn de me lever mais même de décider que je me lèverais.\\
		Suarès \cite{Suares1910} : "Je suis pris au rets de l'or". Idée d'un blocage physique. Image pétrarquisante.\\
		Hesse \cite{Hesse1901} : un gout de Venise pour une certaine molesse [...]"\\
	\subsection{Métaphore du labyrinthe}
		James \cite{James1909} : "vous pouvez y aller chaque jour en décourvrant chaque fois quelque nouvelle image cachée"\\
		Proust EVENTUELLEMENT A COMMENTERPLUS LARGEMENT !!!! \cite{Proust1927} :
		
		
	\subsection{Un attrait paradoxal : la souffrance délicieuse}
		Proust \cite{Proust1927} : le narrateur décide de rester à Venise et dit : "j'assistais à la lente réalisation de mon malheur construit artistement, sans hâte, note par note [...]"
\medskip

\bibliographystyle{frplain}%Used BibTeX style is unsrt
\bibliography{bibliography}

\end{document}
